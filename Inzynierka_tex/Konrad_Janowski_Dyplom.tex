\documentclass[12pt,a4paper]{article}
\usepackage[utf8]{inputenc}
\usepackage{amsmath}
\usepackage{amsfonts}
\usepackage{amssymb}
\usepackage{graphicx}
\usepackage{polski}
\author{Konrad Janowski}
\begin{document}
\section{PN-ISO-1996-1:2006 zdarzenie akustyczne}
\begin{description}
\item[5.1.2] Należy podawać czas trwania zdarzenia w odniesieniu do pewnej cechy dźwięku, jak np, liczba przekroczeń pewnego ustalonego poziomu. 
PRZYKŁAD czas trwania zdarzenia można zdefiniować jako całkowity czas, w którym poziom ciśnienia akustycznego mieści się w zakresie 10 dB maksymalnego poziomu ciśnienia akustycznego podczas zdarzenia.
\end{description}
\end{document}